%%% start of file `template.tex'.
%% Copyright 2006-2010 Xavier Danaux (xdanaux@gmail.com).
%% Copyright 2010-2011 Mark Liu (markwayneliu@gmail.com).
%
% This work may be distributed and/or modified under the
% conditions of the LaTeX Project Public License version 1.3c,
% available at http://www.latex-project.org/lppl/.

\documentclass[10pt,a4paper]{moderncv}
%\usepackage[english]{babel}
\usepackage{verbatim}
\usepackage[margin=10pt,font=small,labelfont=bf]{caption}
% moderncv themes
\moderncvtheme[blue]{classic}                 % optional argument are 'blue' (default), 'orange', 'red', 'green', 'grey' and 'roman' (for roman fonts, instead of sans serif fonts)
%\moderncvtheme[green]{classic}                % idem

% character encoding
\usepackage[utf8]{inputenc}                   % replace by the encoding you are using

% adjust the page margins
\usepackage[scale=0.8]{geometry}
%\setlength{\hintscolumnwidth}{3cm}						% if you want to change the width of the column with the dates
%\AtBeginDocument{\setlength{\maketitlenamewidth}{6cm}}  % only for the classic theme, if you want to change the width of your name placeholder (to leave more space for your address details
%\AtBeginDocument{\recomputelengths}                     % required when changes are made to page layout lengths
%\AtBeginDocument{\setlength{\maketitlenamewidth}{10cm}}
\AtBeginDocument{\recomputelengths}
% personal data
\newcommand*{\skypesymbol} {%
\protect\raisebox{-0.085em}{%
\protect\begin{tikzpicture}[y=0.08em,x=0.08em,xscale=0.022,yscale=-0.022, inner sep=0pt, outer sep=0pt]
 \protect\path[fill=color2,even odd rule] (487.6550,288.9690) .. controls (489.0610,278.5690) and
 (489.8700,267.9960) .. (489.8700,257.2330) .. controls (489.8700,128.0770) and
 (384.5990,23.3610) .. (254.7670,23.3610) .. controls (241.8630,23.3610) and
 (229.2120,24.4210) .. (216.9010,26.4410) .. controls (194.8280,12.0570) and
 (168.5590,3.6740) .. (140.2880,3.6740) .. controls (62.7660,3.6740) and
 (0.0000,66.4820) .. (0.0000,143.9800) .. controls (0.0000,172.1780) and
 (8.2990,198.3740) .. (22.5900,220.3690) .. controls (20.6650,232.3860) and
 (19.6810,244.6920) .. (19.6810,257.2290) .. controls (19.6810,386.4050) and
 (124.8980,491.1100) .. (254.7660,491.1100) .. controls (269.4230,491.1100) and
 (283.6930,489.6840) .. (297.5620,487.1780) .. controls (319.1120,500.5470) and
 (344.4960,508.3260) .. (371.7080,508.3260) .. controls (449.2100,508.3260) and
 (512.0010,445.5020) .. (512.0010,368.0120) .. controls (511.9980,338.7190) and
 (503.0410,311.4840) .. (487.6550,288.9690) -- cycle(276.7400,429.5960) ..
 controls (202.0340,433.4870) and (167.0750,416.9590) .. (135.0500,386.9050) ..
 controls (99.2850,353.3370) and (113.6520,315.0500) .. (142.7900,313.1040) ..
 controls (171.9120,311.1590) and (189.3980,346.1160) .. (204.9410,355.8400) ..
 controls (220.4650,365.5280) and (279.5340,387.6000) .. (310.7350,351.9320) ..
 controls (344.7100,313.1040) and (288.1410,293.0120) .. (246.6760,286.9300) ..
 controls (187.4730,278.1640) and (112.7260,246.1370) .. (118.5410,183.0230) ..
 controls (124.3580,119.9490) and (172.1230,87.6090) .. (222.3910,83.0470) ..
 controls (286.4680,77.2300) and (328.1820,92.7540) .. (361.1760,120.9070) ..
 controls (399.3270,153.4360) and (378.6840,189.8010) .. (354.3770,192.7270) ..
 controls (330.1660,195.6360) and (302.9730,139.2230) .. (249.5860,138.3750) ..
 controls (194.5590,137.5110) and (157.3690,195.6360) .. (225.3000,212.1590) ..
 controls (293.2660,228.6640) and (366.0500,235.4450) .. (392.2610,297.5760) ..
 controls (418.4900,359.7130) and (351.5070,425.7010) .. (276.7400,429.5960) --
 cycle;
  \protect\end{tikzpicture}}%
    ~}
    
\makeatletter
% defines one's email (optional)
% usage: \email{<email adress>}
\newcommand*{\skype}[1]{\def\@skype{#1}}
\renewcommand*{\makecvfooter}{%
  \setlength{\footerwidth}{0.8\textwidth}%
  \fancypagestyle{plain}{%
    \fancyfoot[c]{%
      \parbox[b]{\footerwidth}{%
        \centering%
        \color{color2}\addressfont%
        \vspace{\baselineskip}% forces a white line to ensure space between main text and footer (as footer height can't be known in advance)
            \ifthenelse{\isundefined{\@addressstreet}}{}{\addtofooter[]{\addresssymbol\@addressstreet}%
              \ifthenelse{\equal{\@addresscity}{}}{}{\addtofooter[~--~]{\@addresscity}}% if \addresstreet is defined, \addresscity and \addresscountry will always be defined but could be empty
              \ifthenelse{\equal{\@addresscountry}{}}{}{\addtofooter[~--~]{\@addresscountry}}%
              \flushfooter\@firstfooterelementtrue\\}%
            \collectionloop{phones}{% the key holds the phone type (=symbol command prefix), the item holds the number
              \addtofooter{\csname\collectionloopkey phonesymbol\endcsname\collectionloopitem}}%
            \ifthenelse{\isundefined{\@email}}{}{\addtofooter{\emailsymbol\emaillink{\@email}}}%
            \ifthenelse{\isundefined{\@homepage}}{}{\addtofooter{\homepagesymbol\httplink{\@homepage}}}%
            \ifthenelse{\isundefined{\@skype}}{}{\addtofooter{\skypesymbol\@skype}}%
            \collectionloop{socials}{% the key holds the social type (=symbol command prefix), the item holds the link
              \addtofooter{\csname\collectionloopkey socialsymbol\endcsname\collectionloopitem}}%
            \ifthenelse{\isundefined{\@extrainfo}}{}{\addtofooter{\@extrainfo}}%
            \ifthenelse{\lengthtest{\footerboxwidth=0pt}}{}{\flushfooter}% the lengthtest is required to avoid flushing an empty footer, which could cause a blank line due to the \\ after the address, if no other personal info is used
            }}}%
\pagestyle{plain}}
\makeatother
% personal data
\firstname{Ganesh}
\familyname{H}
\title{Curriculum Vitae}
\address{TC 23/692, Valiyasala,Chalai.P.O,Trivandrum}{Kerala, India - 695036}
%\phone{+91-9446242358}
\mobile{+91-9446242358}
\extrainfo{\skypesymbol ganeshredcobra}
\email{ganeshredcobra@gmail.com}
%\homepage{http://markliu.me}                % optional, remove the line if not wanted
%\extrainfo{\url{http://markliu.me}} % optional, remove the line if not wanted

% to show numerical labels in the bibliography; only useful if you make citations in your resume
%\makeatletter
%\renewcommand*{\bibliographyitemlabel}{\@biblabel{\arabic{enumiv}}}
%\makeatother

%\nopagenumbers{}                             % uncomment to suppress automatic page numbering for CVs longer than one page
%----------------------------------------------------------------------------------
%            content
%----------------------------------------------------------------------------------
\begin{document}
\maketitle
\section{Education}
\cventry{2005--2009}{B.Tech, Applied Electronics \& Instrumentation Engineering}{University of Kerala}{}{}{}
\cvline{College:}{\small P.A.Aziz College of Engineering \& Technology}
\cvline{GPA:}{\small 6.04/10.0}
\cventry{2003--2005}{Higher Secondary}{Biology+Maths Stream}{}{}{}
\cvline{School:}{\small Govt Model Boys H S S Chalai, Trivandrum, Kerala}
\cvline{Percentage:}{\small 75\%}
\cventry{2002--2003}{Secondary School}{Science+Maths+English Stream}{}{}{}  % arguments 3 to 6 can be left empty
\cvline{School:}{\small Govt Model Boys H S S Chalai, Trivandrum, Kerala}
\cvline{Percentage:}{\small 82.50\%}

\section{Work Experience}
\cventry{February 2016 -- Present}{Senior Engineer}{Tata Elxsi}{}{}{
Jaguar and Land Rover Vehicle Program.\\
 \url{http://www.tataelxsi.com 
}   
}
\cventry{}{Job responsiblities}{}{}{}{
Research \& Development of JLR projects.\\
Labcar building and integration testing.\\
Labcar network testing \& Diagnostic testing.\\
}
\cventry{November 2015 -- January 2016}{Senior Software Engineer}{Deep Thought Systems Pvt Ltd}{}{}{
Lead Developer and Designer for Automtoive Embedded System and Embedded Linux applications.\\ \url{http://www.dthoughts.com 
}   
}
\cventry{}{Job responsiblities}{}{}{}{
Responsibilities include performing software requirement analysis.\\
Software Architecture/Design and Software Implementation.\\
Supporting software module \& Integration testing and Software Validation.\\
Supporting customer on investigating \& resolving software issues.\\
}
\cventry{August 2013 -- October 2015}{Embedded Software Engineer}{Deep Thought Systems Pvt Ltd}{}{}{
Design,develop \& debug embedded firmware and software for Microcontrollers.Design \& develop projects for Embedded Automotive and Embedded Linux platforms.\\ \url{http://www.dthoughts.com 
}  
}
\cventry{}{Job responsiblities}{}{}{}{
Embedded Firmware \& Software programming and test on microcontroller-based products.\\
Single board Computer \& Embedded Linux based development.\\
Interfacing of sensor modules to microcontrollers.\\
Research \& Development of company products.\\
Design \& develop projects on Embedded Automotive domain .\\
}
\cventry{February 2010 -- July 2013}{Software Engineer}{Society for Promotion of Alternative Computing and Employment}{}{}{
Developed and implemented many applications using various free software tools.Conducted research on using free software tools for scientific computing,Embedded Linux \& System Administration.\\ \url{http://space-kerala.org 
}  
}
\cventry{}{Job responsiblities}{}{}{}{
Customization, deployment and maintenance of Linux distribution on different hardware
platforms.\\
Embedded C programming, integration and test on microcontroller-based products.\\
Interact with Linux kernel and integrate new device drivers.\\
Cross compile Linux kernel for multiple platforms.\\
%Serve as the primary systems administrator for Linux servers and services.\\
Develop python scripts for server side automation \& testing.\\
Conduct research on open hardware \& FOSS in Science  Engineering.\\
}
%\cventry{February 2010 -- present}{Freelance Hacker}{}{}{}{
%A Freelance programmer and consultant on Free and Open Source Software based %technologies(GNU/Linux,Python,Embedded Linux etc).}
\cventry{July 2009 -- January 2010}{Lecturer cum Support Engineer}{                                            IRS Informatics India pvt Ltd}{}{}{
Worked as Lecturer cum Support Engineer in Department of Instrumentation and Automation of IRS Informatics India pvt Ltd, Thrissur from July 2009 to January 2010,in their projects for C-DIT and KELTRON.Works include conduct theory and practical classes for the course Diploma in Computerised Instrumentation.
} 


\section{Technical Skills}
%\subsection{Have Experience With}
\cvline{\textbf{Programming Languages}}{\hspace{0.1cm}Assembly, Embedded C, Python, HTML/CSS, Bash Scripting, Processing.}%,Matlab
\cvline{\textbf{Operating Systems}}{Proficient in Debian,Ubuntu/Kubuntu Linux and other Linux variants, Windows 9x, XP, Vista, Virtualization of Windows \& Linux guests using Sun VirtualBox.}
%\cvline{\textbf{Authoring \& Graphics}}{\LaTeX\ ,\LaTeX Beamer ,Open Office , GIMP.}
%\cvline{\textbf{Microcontroller's \& Dev Boards}}{PIC 16F877A,TI MSP430,STM32F4Discovery,Mini2440(Arm 9),AtmelAVR.}
\cvline{\textbf{Controller's}}
{NXP - LPC 17XX Series. \newline
FreeScale - i.MX 28 Series, MPC 56XX Series. \newline
Microchip - PIC 16F877A,PIC 18F25K80, MCP 2515.\newline
Atmel AVR - ATmega8,ATmega168,ATmega328,tinyAVR.\newline
Texas Instruments - MSP430 Launchpad,MSP430F5438A,C2000,Tiva C Series.
}
\cvline{\textbf{SBC's}}{Raspberry Pi, BeagleBone Black, Wandboard.}
\cvline{\textbf{IDE's}}{Eclipse,Code Composer Studio,MPLab,Keil,Code Warrior.}
\cvline{\textbf{Development Tools}}{GNU Tools, Git, SVN, Vim.}
\cvline{\textbf{Debuggers}}{Lauterbach Trace 32, ULink2ME, PE Micro, ST Link, PICKit.}
\cvline{\textbf{CAN Tools}}{VECTOR CANalyzer, NeoVI, Kvaser, Microchip CAN Analyzer, CANMATE.}
\cvline{\textbf{Protocols}}{CAN, I2C, SPI, RS232.}
\cvline{\textbf{Standards}}{ANSI C, OBDII, J1979, J1939, UDS.}
%\cvline{\textbf{Circuit/PCB Design}}{Eagle PCB,Kicad.}
%\cvline{\textbf{PLC's Used}}{Siemens-S7300(Simatic Manager),Siemens-S7200(Step7 Microwin),AB Micrologix 1000(RSLogix 500).}
%\cvline{\textbf{HMI}}{Siemens TP177B(Touch Panel).}


\section{Project Works}
\cvline{Tata Elxsi}{\textbf{Anti Pinch for Power Window} - Developed and implemented Anti Pinching algorithm for power window system. The project uses MPC5604 as its controller. The project make use of current sensing, Hall sensors, CAN and LIN bus messages.}
\cvline{Dthoughts Systems}{\textbf{Vehicle specific firmware development / Onsite Dev} - This was an Onsite assignment.The client is a famous automotive after market perfomance and race products company at California,USA.The task were to reverse engineer factory scantool and find vehicle specific pids and add that PID's to their existing firmware ,Add J1939 support to their existing firmware, Solve Dynamic PGN and add UDS support to the firmware.The controller used is LPC1768.}
\cvline{Dthoughts Systems}{\textbf{CANMate Linux} - CANMATE is a high performing CAN bus analyzer suitable to sniff any CAN network operating at or near full load and at full speed. Any CAN network can be analyzed using CANMATE device with the help of associated GUI tools. It can be also used as a CAN node simulator. Was key player in developing GNU/Linux version of CANMate.Developed Linux packages and shared objects for the project.The product uses PIC18F25K80 controller.\newline
\textbf{Product Page : }\url{http://www.dthoughts.com/products/canmate.html}
}
\cvline{Dthoughts Systems}{\textbf{ECU Simulator{(J1939 \& J1979 Simulator)} } - ECU Simulator is a small, lightweight, OBD bench simulator for OBD hardware and software development and testing.It supports both SAE J1939 and SAE J1979.This simulator has 5 user-adjustable PIDs using 5 potentiometer , 5 live data can be changed and 6 fixed values are provided. Pressing DTC button the trouble codes are generated and the MIL LED is ON.J1979 mode support both 11/29 bit,VIN,DTC and also supports Three default virtual ECU's: ECM, TCM, and ABS. J1939 mode supports DM1,DM5 and DM11 Diagnostic messages. Developed Hardware and Firmware for the project.
\textbf{Product Page : }\url{http://www.dthoughts.com/products/combosim1000.html}
}
\cvline{Dthoughts Systems}{\textbf{ECU Simulator Linux} - ECUSimLite Linux is a free PC application which uses CANMate device and CANMate API to implement ISO15765 compliant OBD2 ECU simulation.This can be used to test devices like OBD port readers without the need to connect to a vehicle.Supports Mode 01,Mode 03 and Mode 04. Five user variable parameters are available, supports Set \& Clear DTC along with logging facility. Completely coded in gcc.\newline
\textbf{Product Page : }\url{http://www.dthoughts.com/products/ecusimulator.html}
}
\cvline{Dthoughts Systems}{\textbf{J1939 Simulator Linux} - J1939ECUSim is a PC application which uses our CANMate device and CANMate API to implement J1939 compliant ECU simulation.This simulator has 5 user-adjustable PGNs. The DTC button can be used to generate trouble codes. Some of SAE J1939 compliant DMs are implemented. Completely coded in gcc.\newline
\textbf{Product Page : }\url{http://www.dthoughts.com/products/j1939ecusim.html}
}
\cvline{Dthoughts Systems}{\textbf{AMT CAN Log Analyzer} - Automated Manual Transmission(AMT) CAN Log Analyzer is used to find out errors from Engine,Brake and Transmission Unit for J1939 Vehicles.The project uses PIC18F25K80 controller and MCP 2551 CAN transceiver.Keeping standard J1939 message transmission rates as reference the project finds out the missing messages from specific modules like ABS,Cruise Control etc.The CAN messages are logged and mssing messages are reported in software user interface.}
\cvline{Dthoughts Systems}{\textbf{CAN-DBC parser} - The aim of the project is to parse the DBC files used by Vector. The project uses cantools which is a set of command line tools for dealing with DBC, ASC,and MDF files. The tools can be used to analyze and convert the data to other formats. The project is completely coded in gcc. The project is an add-on feature for CANMate low cost CAN analyzer.}
\cvline{Dthoughts Systems}{\textbf{J1939 DAQ System} - This project implements a high speed data acquisition system. Five analog sensors can be connected to the DAQ system. The sensor values are packed into the CAN message and transmitted over the CAN bus. The data is parsed and shown in a GUI interface on the system side and is logged into csv sheet. The project implements J1939 broadcast messaging system. The project used PIC 18F25K80 microcontroller.}
\cvline{Dthoughts Systems}{\textbf{Bluetooth Low Energy Project} - Implemented automatic SPPLE Server based pairing and communication between MSP430F5438A along with PAN 1323 BLE Module and smart phone.The project uses Bluetopia bluetooth stack by Stonestreet One.The project uses MSP-FET430UIF which is used to program and debug MSP430 FET tools and test boards through the JTAG interface.}
\cvline{Dthoughts Systems}{\textbf{BSP porting and customization of LINUX on i.MX28 processor} - The aim of this project is to port and customize Board Support Package of Linux operating system on FreeScale i.MX28 processor based embedded board with the help of Linux Target Image Builder to configure WiFi and Bluetooth. The WiFi and Bluetooth driver is compiled with cross-compiler that released with i.mx28 linux BSP(LTIB) and installed its firmware.The wireless tools,bluetooth commands are used to configure WiFi and Bluetooth on board.Configured Linux Socket CAN interface to send and receive CAN messages.}
\cvline{Dthoughts Systems}{\textbf{Streaming server in Beagle Bone Black} - Converting Beagle Bone Balck to video streaming server.The project uses MJPG streamer. MJPG-streamer takes JPGs from Linux-UVC compatible webcams, filesystem or other input plugins and streams them as M-JPEG via HTTP to webbrowsers, VLC and other software.}
%\subsection{Academic Projects}
\cvline{SPACE}{\textbf{Raspberry Pi based Remote Weather station} - The aim of project is to buils a remote weather station which can update details to a cloud platform where it can be plotted.In this project various sensors to measure temperature,relative humidity,rain and soil moisture is added.The microcontroller used is PIC16F877A.The controller board is connected to a Raspberry Pi via USBserial.The data is logged to a remote cloud server using python.}
\cvline{SPACE}{\textbf{Porting and board bring up of Mini2440} - The aim of this project is to port and customize Board Support Package of Linux operating system on Mini2440(Arm9) development board.Worked on software and platform support for a FriendlyARM project (including porting qtopia, QT, Python, and transforming other utilities from full-scale to embedded) on the mini2440 board. Successfully ported multiple open-source tools to the Mini2440.Wrote Linux user space programs to access GPIO and ADC in Mini2440.}
\cvline{SPACE}{\textbf{Security System} -It is having various sensors incorporated in it for the detection of intruders and some sensors to sense some natural mishaps like fire etc. It is having a temperature sensor(LM 35), a smoke sensor and a motion detector(PIR Sensor). A receiver is there to show the signal if anyone of the sensor is active it can be taken to a distant place according to the power of the transmitter and receiver(TWS \& RWS 434).This project uses PIC 16F877A Controller.}
%\cvline{Main Project Image Fusion/Mosaicing}{Many problems require finding the coordinate transformation %between two images of the same scene or object. One of them is Image Mosaicing/ Image warping. It is %important to have a precise description of the coordinate transformation between a pair of images. Image %mosaics are collection of overlapping images together with coordinate transformations that relate the %different image coordinate systems. By applying the appropriate transformations via a warping operation and %merging the overlapping regions of a warped image, it is possible to construct a single image covering the %entire visible area of the scene. This merged single image is the motivation for the term ``mosaic''. The %project is  implemented in MATLAB.}
%\subsection{Freelance Electronics/Robotics Projects}
\cvline{SPACE}{\textbf{RFID Based Access Control Sysytem} -  This project uses PIC 16F877A Controller and a pre-built RFID reader module to interrogate commonly-available passive tags, looks up the tag ID in an internal database, and release a lock if the tag is authorised.}
%\cvline{}{\textbf{Gesture Controlled TV} - In this project a kinect  is used.With the help of processing software gestures are recognised from kinect.For the correct gesture an appropriate remote signal is send using IR Led.}
\cvline{SPACE}{\textbf{Email Controlled Home Appliances}-The project helps to control home appliances using email.The project uses Arduino board and a python script which helps to fetch mail and reply.} 
%\cvline{SPACE}{\textbf{Gesture Controlled ROBOT} - In this project a kinect  is used.With the help of processing software gestures are recognised from kinect.According to the correct gesture the robot moves.}
%\cvline{}{\textbf{Intelligent Display Board} - In this project the display board lights up and shutdown on correct time automatically.The projects uses a DS1307 RTC for automatic wakeup and shutdown.}
\cvline{SPACE}{\textbf{Converting wall to a touch screen} - This project uses Kinect camera to turn different surfaces to multitouch surfaces.Using LKB bundle kinect camera is calibrated and TUIO Signals are got at the output.A python script is used to convert the TUIO signals to mouse events.}
\cvline{SPACE}{\textbf{SMS based device control} - This project uses PIC 16F877A along with a GSM modem.The project helps to control home appliances using SMS.}
\cvline{SPACE}{\textbf{Disaster Management} -This project uses PIC 16F877A along with a GSM modem and some sensors like LM35 and MQ5 Gas sensor.When a mishap is sensed an SMS will be sent to the preset number.}
\cvline{Hobby}{\textbf{Fleet Vehicle Tracking (OnGoing)} - OBDSam is a hobby project developed for Fleet vehicle tracking. The product collects vehicle information such as RPM, APP, ECT, Speed,Fuel Level etc from OBD port and also adds GPS coordinates and updates to web dashboard using GPRS connection. The web dashboard is developed in Flask , a python web framework.It also collect's routine DTC information.The project allows vehicle tracking and vehicle health monitoring.The project is developed in LPC1768.\newline
\textbf{Product Page : }\url{http://obdsam.herokuapp.com}
}
%\cvline{}{\textbf{Water Tank Level Indicating and Controlling System using LDR} - had two special features that other level controlling systems in the market lacked – low cost implementation and elimination of all kind of immersions into the water. The op-amp IC LM324 and a LDR were used to realize the circuit.}
%\cvline{}{\textbf{Arduino Voltmeter} - Arduino board uses ATmega series of Microcontrollers of Atmel AVR Family. The range of the voltmeter will be between 0 and 10 volts DC – ideal for testing batteries and cells before they head to the garbage bin.}
%\cvline{}{\textbf{Light Chasing Robot} – A simple robot that follows the light (in a dark room). It used LDR as the light sensor.}
%\cvline{}{\textbf{Line Following Robot} – A highly useful robot that could follow the line drawn on a surface. An IR LED was the sensor used.}
%\cvline{}{\textbf{PC Controlled Robot} – A robot that could be navigated through a particular path using a computer. It worked by processing the data received from a computer’s serial port with microcontroller (Arduino).}
%\cvline{}{\textbf{Obstacle Avoiding Robot} – This robot could take a deviation in its path whenever it detected any obstacle that blocked its path. IR LED was used to detect the obstacle.}
%\cvline{}{\textbf{Sound Operated Robot} – A robot that responds to the sound (eg: a clap). It is programmed to deviate its path when it detects a sound signal. The sensor used was a normal microphone.}

\subsection{Software Projects}
\cvline{}{Customisation of  Python based software for conducting IT examination for IT@SCHOOL.} 
\cvline{}{Various scripts (Python\& shell) for System Administration \& Embedded Automation.}
\cvline{}{Linux OS Customization (Remastersys \& commandline)}
\cvline{}{Hacking Microsoft Kinect using Processing and libfreenect libraries.}
\cvline{}{Device Drivers \& Kernel Programming (Beginner)}
\cvline{}{Linux system Programming(Process,Signals,Threads).}
%\cvline{}{}


\section{Trainings \& Workshops Undergone}
\cvline{}{Attended training by SIEMENS on Simatic S7-300. The course included S7-300 series PLC’s programming, communication, connections, interfacing etc. Also a touch screen HMI panel – TP177B.}
\cvline{}{Attended a workshop on Microcontroller based Robotics (level III) conducted by LI2 Innovations, Bangalore and Govt: Engineering College, Thrissur.(Arduino)}
\cvline{}{Attended TI MCU DeepDive 2013 conducted by Texas Instruments India.}

\section{Achievements \& Activities}
\cvline{}{Key Organiser Hardware Freedom Day 2013,Kerala.}
\cvline{}{Active member of the FSUG (Free Software Users Group) TVPM.}
\cvline{}{Program management team member of International Conference on SCIPY (Scientific Python).}
\cvline{}{Participated in International Conference on Sagemath held at IIT-Bombay.}
%\cvline{}{Participated in Open Solaris workshop held at College of Engineering Trivandrum.}
\cvline{}{Got media coverage and appreciation for developing kinect based apps for autistic kids.}

\section{Talks \& Workshops Conducted}
\cventry{Februray 2014}{Introduction to Open Hardware}{National Institute of Technology , Calicut}{}{}{Handson workshop on Arduino and Processing.}
\cventry{September 2012}{Training of Assistive Technologies}{Jawaharlal Nehru University,New Delhi}{}{}{Workshop on Open Accessibility and assistive technology.}
\cventry{August 2012}{FOSS tools for science and engineering.}{SB College Changanacherry}{}{}{Introduction to open hardware and other open source scientific computing packages.}
\cventry{June 2012}{\LaTeX Workshop}{Amal Jyothi College of Engineering,Kanjirapally}{}{}{\LaTeX workshop for  Faculties from AICTE, UGC approved Institutions, researchers from industries and PG scholars.}
\cventry{May 2012}{Linux System Administartion Training.}{Institute of Management in Government (IMG), Trivandrum}{}{}{Introduction to GNU/Linux System Administration.}
\cventry{January 2012}{Introduction To Python And FOSS Technologies.}{Govt. Polytechnic, Trikaripur}{}{}{Python workshop,Python's capabilities and alternatives in FOSS against proprietary software}
\cventry{November 2011}{Workshop on FOSS Tools  for research scholars.}{Government College of Teacher Education, Thycaud, Thiruvananthapuram}{}{}{Workshop on FOSS tools for PhD research scholars.}
%\cventry{September 2012}{Introduction To Webdevelopment using FOSS.}{Younus College of Engineering and Technology,Kollam}{}{}{Introduction to web development using FOSS tools such as Wordpress and Drupal etc.}

\section{Interests \& Hobbies.}
\cvline{}{An avid reader on technology, fiction and non-fiction books and websites.}
\cvline{}{Making Hobby circuits.}
\cvline{}{Trouble shooting problems in computers \& Ubuntu Linux.}
\cvline{}{Scripting using Python and Bash.}

\section{Blog \& Code.}
%\begin{minipage}{.5\linewidth}
%  \centering
\cvline{\includegraphics[width=30mm]{github}}{\url{https://github.com/ganeshredcobra}}
%\cvline{}{}
\cvline{\includegraphics[width=10mm]{wordpress}}{\url{http://importgeek.wordpress.com/}}  
%  \captionof{figure}{your caption}
%\end{minipage}

\end{document}
