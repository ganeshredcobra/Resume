%%% start of file `template.tex'.
%% Copyright 2006-2010 Xavier Danaux (xdanaux@gmail.com).
%% Copyright 2010-2011 Mark Liu (markwayneliu@gmail.com).
%
% This work may be distributed and/or modified under the
% conditions of the LaTeX Project Public License version 1.3c,
% available at http://www.latex-project.org/lppl/.

\documentclass[10pt,a4paper]{moderncv}
\usepackage{verbatim}
\usepackage[margin=10pt,font=small,labelfont=bf]{caption}

% moderncv themes
\moderncvtheme[blue]{classic}                 % optional argument are 'blue' (default), 'orange', 'red', 'green', 'grey' and 'roman' (for roman fonts, instead of sans serif fonts)
%\moderncvtheme[green]{classic}                % idem

% character encoding
\usepackage[utf8]{inputenc}                   % replace by the encoding you are using

% adjust the page margins
\usepackage[scale=0.8]{geometry}
%\setlength{\hintscolumnwidth}{3cm}						% if you want to change the width of the column with the dates
%\AtBeginDocument{\setlength{\maketitlenamewidth}{6cm}}  % only for the classic theme, if you want to change the width of your name placeholder (to leave more space for your address details
%\AtBeginDocument{\recomputelengths}                     % required when changes are made to page layout lengths
%\AtBeginDocument{\setlength{\maketitlenamewidth}{10cm}}
\AtBeginDocument{\recomputelengths}
% personal data
% personal data
\firstname{Ganesh}
\familyname{H}
\title{Curriculum Vitae}
\address{TC 23/692, Valiyasala,
Chalai.P.O,Trivandrum}{Kerala, India - 695036}
%\phone{+91-9446242358}
\mobile{+91-9446242358}
\email{ganeshredcobra@gmail.com}
%\homepage{http://markliu.me}                % optional, remove the line if not wanted
%\extrainfo{\url{http://markliu.me}} % optional, remove the line if not wanted

% to show numerical labels in the bibliography; only useful if you make citations in your resume
%\makeatletter
%\renewcommand*{\bibliographyitemlabel}{\@biblabel{\arabic{enumiv}}}
%\makeatother

%\nopagenumbers{}                             % uncomment to suppress automatic page numbering for CVs longer than one page
%----------------------------------------------------------------------------------
%            content
%----------------------------------------------------------------------------------
\begin{document}
\maketitle

\section{Education}
\cventry{2005--2009}{B.Tech, Applied Electronics \& Instrumentation Engineering}{University of Kerala}{}{}{}
\cvline{College:}{\small P.A.Aziz College of Engineering \& Technology}
\cvline{gpa:}{\small 6.04/10.0}
\cventry{2003--2005}{Higher Secondary}{Biology+Maths Stream}{}{}{}
\cvline{School:}{\small Govt Model Boys H S S Chalai, Trivandrum, Kerala}
\cvline{gpa:}{\small 7.5/10.0}
\cventry{2002--2003}{Secondary School}{Science+Maths+English Stream}{}{}{}  % arguments 3 to 6 can be left empty
\cvline{School:}{\small Govt Model Boys H S S Chalai, Trivandrum, Kerala}
\cvline{gpa:}{\small 8.25/10.0}

\section{Work Experience}
\cventry{August 2013 -- Present}{Embedded Software Engineer}{Deep Thought Systems Pvt Ltd}{}{}{
Design,develop \& debug embedded firmware and software for 8,16,32 bit Microcontrollers.Design \& develop projects for Arm and embedded Linux platforms.\\ \url{http://www.dthoughts.com 
}  
}


\cventry{}{Job responsiblities}{}{}{}{
Embedded Firmware \& Software programming and test on microcontroller-based products.\\
Single board Computer \& Embedded Linux based development.\\
Interfacing of sensor modules to microcontrollers.\\
Research \& Development of company products.\\
Design \& develop projects on ARM Series .\\
}
\cventry{February 2010 -- July 2013}{Software Engineer}{Society for Promotion of Alternative Computing and Employment}{}{}{
Developed and implemented many applications using various free software tools.Conducted research on using free software tools for scientific computing,Embedded Linux \& System Administration.\\ \url{http://space-kerala.org 
}  
}


\cventry{}{Job responsiblities}{}{}{}{
Customization, deployment and maintenance of Linux distribution on different hardware
platforms.\\
Embedded C programming, integration and test on microcontroller-based products.\\
Interact with Linux kernel and integrate new device drivers.\\
Cross compile Linux kernel for multiple platforms.\\
Serve as the primary systems administrator for Linux servers and services.\\
Develop python scripts for server side automation \& testing.\\
Conduct research on open hardware \& FOSS in Science  Engineering.\\
}
%\cventry{February 2010 -- present}{Freelance Hacker}{}{}{}{
%A Freelance programmer and consultant on Free and Open Source Software based %technologies(GNU/Linux,Python,Embedded Linux etc).}
\cventry{July 2009 -- January 2010}{Lecturer cum Support Engineer}{                                            IRS Informatics India pvt Ltd}{}{}{
Worked as Lecturer cum Support Engineer in Department of Instrumentation and Automation of IRS Informatics India pvt Ltd, Thrissur from July 2009 to January 2010,in their projects for C-DIT and KELTRON.Works include conduct theory and practical classes for the course Diploma in Computerised Instrumentation.
} 

\newpage{}
\section{Technical Skills}
%\subsection{Have Experience With}
\cvline{\textbf{Programming Languages}}{C,Assembly,Embedded C,Python,HTML/CSS,Bash Scripting,Processing.}%,Matlab
\cvline{\textbf{Operating Systems}}{Proficient in Debian,Ubuntu/Kubuntu Linux and other Linux variants, Windows 9x, XP, Vista, Virtualization of Windows \& Linux guests using Sun VirtualBox.Have good experience in GNU/Linux system administration and Shell scripting(BASH).}
\cvline{\textbf{Authoring \& Graphics}}{\LaTeX\ ,\LaTeX Beamer ,Open Office , GIMP.}
%\cvline{\textbf{Microcontroller's \& Dev Boards}}{PIC 16F877A,TI MSP430,STM32F4Discovery,Mini2440(Arm 9),AtmelAVR.}
\cvline{\textbf{Microcontroller's}}
{ Microchip - PIC 16F877A,PIC 18F25K80.\newline
Atmel AVR - ATmega8,ATmega168,ATmega328,tinyAVR.\newline
Texas Instruments - MSP430 Launchpad,MSP430F5438A,C2000,Tiva C Series(ARM Cortex).
}
\cvline{\textbf{IDE's Used}}{Eclipse,Code Composer Studio,MPLab,Keil,Arduino.}
\cvline{\textbf{Development Tools used}}{GNU Tools,Git,Vim.}
\cvline{\textbf{Protocols}}{I2C, SPI, RS232.}%Removed CAN%
\cvline{\textbf{Circuit/PCB Design}}{Eagle PCB,Kicad.}
\cvline{\textbf{PLC's Used}}{Siemens -S7300(Simatic Manager),Siemens -S7200(Step7 Microwin),AB Micrologix 1000(RSLogix 500).}
\cvline{\textbf{HMI}}{Siemens TP177B(Touch Panel).}


\section{Project Works}
\cvline{Dthoughts Systems}{\textbf{CANMate Linux} - CANMATE is a high performing , low cost CAN to USB converter suitable to sniff any CAN network operating at or near full load and at full speed. Any CAN network can me analyzed using CANMATE device with the help of associated GUI tools.Was key player in developing GNU/Linux version of CANMate.Developed Linux packages and shared objects for the project.The product uses PIC18F25K80 controller}
\cvline{Dthoughts Systems}{\textbf{Bluetooth Low Energy Project} - Implemented automatic SPPLE Server based pairing and communication between MSP430F5438A along with PAN 1323 BLE Module and smart phone.The project uses Bluetopia bluetooth stack by Stonestreet One.The project uses MSP-FET430UIF which is used to program and debug MSP430 FET tools and test boards through the JTAG interface.}
%\subsection{Academic Projects}
\cvline{SPACE}{\textbf{Home Security System} -It is having various sensors incorporated in it for the detection of intruders and some sensors to sense some natural mishaps like fire etc. It is having a temperature sensor(LM 35), a smoke sensor and a motion detector(PIR Sensor). A receiver is there to show the signal if anyone of the sensor is active it can be taken to a distant place according to the power of the transmitter and receiver(TWS \& RWS 434).This project uses PIC 16F877A Controller.}
%\cvline{Main Project Image Fusion/Mosaicing}{Many problems require finding the coordinate transformation %between two images of the same scene or object. One of them is Image Mosaicing/ Image warping. It is %important to have a precise description of the coordinate transformation between a pair of images. Image %mosaics are collection of overlapping images together with coordinate transformations that relate the %different image coordinate systems. By applying the appropriate transformations via a warping operation and %merging the overlapping regions of a warped image, it is possible to construct a single image covering the %entire visible area of the scene. This merged single image is the motivation for the term ``mosaic''. The %project is  implemented in MATLAB.}
%\subsection{Freelance Electronics/Robotics Projects}
\cvline{SPACE}{\textbf{Raspberry Pi based Remote Weather station} - In this project various sensors to measure temperature,relative humidity,rain and soil moisture is added.The microcontroller used is PIC16F877A.The controller board is connected to a Raspberry Pi via USBserial.The data is logged to a remote cloud server using python.}
\cvline{SPACE}{\textbf{RFID Based Access Control Sysytem} -  This project uses PIC 16F877A Controller and a pre-built RFID reader module to interrogate commonly-available passive tags, looks up the tag ID in an internal database, and release a lock if the tag is authorised.}
%\cvline{}{\textbf{Gesture Controlled TV} - In this project a kinect  is used.With the help of processing software gestures are recognised from kinect.For the correct gesture an appropriate remote signal is send using IR Led.}
\cvline{SPACE}{\textbf{Email Controlled Home Appliances}-The project helps to control home appliances using email.The project uses Arduino board and a python script which helps to fetch mail and reply.} 
\cvline{SPACE}{\textbf{Gesture Controlled ROBOT} - In this project a kinect  is used.With the help of processing software gestures are recognised from kinect.According to the correct gesture the robot moves.}
%\cvline{}{\textbf{Intelligent Display Board} - In this project the display board lights up and shutdown on correct time automatically.The projects uses a DS1307 RTC for automatic wakeup and shutdown.}
\cvline{SPACE}{\textbf{Converting wall to a touch screen} - This project uses Kinect camera to turn different surfaces to multitouch surfaces.Using LKB bundle kinect camera is calibrated and TUIO Signals are got at the output.A python script is used to convert the TUIO signals to mouse events.}
\cvline{SPACE}{\textbf{SMS based device control} - This project uses PIC 16F877A along with a GSM modem.The project helps to control home appliances using SMS.}
\cvline{SPACE}{\textbf{Disaster Management} -This project uses PIC 16F877A along with a GSM modem and some sensors like LM35 and MQ5 Gas sensor.When a mishap is sensed an SMS will be sent to the preset number.}
%\cvline{}{\textbf{Water Tank Level Indicating and Controlling System using LDR} - had two special features that other level controlling systems in the market lacked – low cost implementation and elimination of all kind of immersions into the water. The op-amp IC LM324 and a LDR were used to realize the circuit.}
%\cvline{}{\textbf{Arduino Voltmeter} - Arduino board uses ATmega series of Microcontrollers of Atmel AVR Family. The range of the voltmeter will be between 0 and 10 volts DC – ideal for testing batteries and cells before they head to the garbage bin.}
%\cvline{}{\textbf{Light Chasing Robot} – A simple robot that follows the light (in a dark room). It used LDR as the light sensor.}
%\cvline{}{\textbf{Line Following Robot} – A highly useful robot that could follow the line drawn on a surface. An IR LED was the sensor used.}
%\cvline{}{\textbf{PC Controlled Robot} – A robot that could be navigated through a particular path using a computer. It worked by processing the data received from a computer’s serial port with microcontroller (Arduino).}
%\cvline{}{\textbf{Obstacle Avoiding Robot} – This robot could take a deviation in its path whenever it detected any obstacle that blocked its path. IR LED was used to detect the obstacle.}
%\cvline{}{\textbf{Sound Operated Robot} – A robot that responds to the sound (eg: a clap). It is programmed to deviate its path when it detects a sound signal. The sensor used was a normal microphone.}

\subsection{Software Projects}
\cvline{}{Customisation of  Python based software for conducting IT examination for IT@SCHOOL.} 
\cvline{}{Various scripts (Python\& shell) for System Administration.}
\cvline{}{Linux OS Customization (Remastersys \& commandline)}
\cvline{}{Hacking Microsoft Kinect using Processing and libfreenect libraries.}
\cvline{}{Device Drivers \& Kernel Programming (Beginner)}
\cvline{}{Linux system Programming(Process,Signals,Threads).}
%\cvline{}{}


\section{Trainings \& Workshops Undergone}
\cvline{}{Attended training by SIEMENS on Simatic S7-300. The course included S7-300 series PLC’s programming, communication, connections, interfacing etc. Also a touch screen HMI panel – TP177B.}
\cvline{}{Attended a workshop on Microcontroller based Robotics (level III) conducted by LI2 Innovations, Bangalore and Govt: Engineering College, Thrissur.(Arduino)}
\cvline{}{Attended Robotics workshops conducted by IEEE.}

\section{Achievements \& Activities}
\cvline{}{Key Organiser Hardware Freedom Day 2013,Kerala.}
\cvline{}{Active member of the FSUG (Free Software Users Group) TVPM.}
\cvline{}{Program management team member of International Conference on SCIPY (Scientific Python).}
\cvline{}{Participated in International Conference on Sagemath held at IIT-Bombay.}
%\cvline{}{Participated in Open Solaris workshop held at College of Engineering Trivandrum.}
\cvline{}{Got media coverage and appreciation for developing kinect based apps for autistic kids.}

\section{Talks \& Workshops Conducted}
\cventry{September 2012}{Training of Assistive Technologies}{Jawaharlal Nehru University,New Delhi}{}{}{Workshop on Open Accessibility and assistive technology.}
\cventry{August 2012}{FOSS tools for science and engineering.}{SB College Changanacherry}{}{}{Introduction to open hardware and other open source scientific computing packages.}
\cventry{June 2012}{\LaTeX Workshop}{Amal Jyothi College of Engineering,Kanjirapally}{}{}{\LaTeX workshop for  Faculties from AICTE, UGC approved Institutions, researchers from industries and PG scholars.}
\cventry{May 2012}{Linux System Administartion Training.}{Institute of Management in Government (IMG), Trivandrum}{}{}{Introduction to GNU/Linux System Administration.}
\cventry{January 2012}{Introduction To Python And FOSS Technologies.}{Govt. Polytechnic, Trikaripur}{}{}{Python workshop,Python's capabilities and alternatives in FOSS against proprietary software}
\cventry{November 2011}{Workshop on FOSS Tools  for research scholars.}{Government College of Teacher Education, Thycaud, Thiruvananthapuram}{}{}{Workshop on FOSS tools for PhD research scholars.}
%\cventry{September 2012}{Introduction To Webdevelopment using FOSS.}{Younus College of Engineering and Technology,Kollam}{}{}{Introduction to web development using FOSS tools such as Wordpress and Drupal etc.}

\section{Interests \& Hobbies.}
\cvline{}{An avid reader on technology, fiction and non-fiction books and websites.}
\cvline{}{Making Hobby circuits.}
\cvline{}{Trouble shooting problems in computers \& Ubuntu Linux.}
\cvline{}{Scripting using Python and Bash.}

\section{Blog \& Code.}
%\begin{minipage}{.5\linewidth}
%  \centering
\cvline{\includegraphics[width=30mm]{github}}{\url{https://github.com/ganeshredcobra}}
%\cvline{}{}
\cvline{\includegraphics[width=10mm]{wordpress}}{\url{http://importgeek.wordpress.com/}}  
%  \captionof{figure}{your caption}
%\end{minipage}

\end{document}
